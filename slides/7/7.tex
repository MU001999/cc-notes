\documentclass[UTF8,aspectratio=169,mathserif]{beamer}
\usepackage{ctex}
\usepackage{ulem}
\usepackage{color}
\usepackage{amssymb}
\usepackage{amsmath}
\usetheme{Berlin}
\setbeamertemplate{navigation symbols}{}

\title{7 - 随机计算}
\subtitle{Randomized computation}
\author{报告人:许博}
\date{2021年5月12日}

\begin{document}
	
	\begin{frame}
		\titlepage
	\end{frame}

	\begin{frame}{目录}
		\tableofcontents
	\end{frame}

	\section{概率图灵机和类$\bf BPP$}
	\begin{frame}{随机算法}{Randomized Algorithm}
		随机算法是可能涉及随机选择的算法。\newline
		
		实际中,随机算法使用随机数生成器实现,\newline 而事实上一个能够投掷公平硬币的生成器就足够了。
	\end{frame}
	
	\begin{frame}{概率图灵机}{Probabilistic Turing Machines}
		\includegraphics[width=\linewidth]{../../7/note.assets/image-20210508132612511.png}\newline
		
		PTM 和 NDTM 句法相似( syntactically similar),但概念上更像 DTM。
	\end{frame}
	
	\begin{frame}{类$\bf BPP$}{\textbf{B}ounded-error \textbf{P}robabilistic \textbf{P}olynomial-time}
		对 $L\subseteq\{0,1\}^*$ 和 $x\in\{0,1\}^*$,定义 $L(x)=1$ 如果 $x\in L$,否则 $L(x)=0$。\newline
		
		\includegraphics[width=\linewidth]{../../7/note.assets/image-20210508132647720.png}\newline
		
		$\bf BPP$ 类似 $\bf P$,仍然限制最坏情况下的复杂性。有$\bf P\subseteq BPP$。
	\end{frame}

	\begin{frame}{使用 DTM 定义 $\bf BPP$}
		\includegraphics[width=\linewidth]{../../7/note.assets/image-20210508142316582.png}\newline
		
		有 $\bf BPP\subseteq EXP$。\newline
		
		${\bf BPP}=?\ {\bf P}$
	\end{frame}
	
	\section{一些 PTM 的例子}
	\begin{frame}{查找第 $k$ 小的数}{Finding the Kth Smallest
			Number}
	\end{frame}

	\begin{frame}{概率素数检测}{Probabilistic Primality Testing}
	\end{frame}
	
	\begin{frame}{多项式恒等检测}{Polynomial Identity Testing}
	\end{frame}

	\begin{frame}{二分图完备匹配检测}{Testing for Perfect Matching in a Bipartite Graph}
	\end{frame}
	
	\section{单边误差和“零边”误差}
	\begin{frame}{双边误差}{Two-sided Error}
		双边误差的概率算法即允许用于语言 $L$ 的算法在 $x\in L$ 时输出 0 以及在 $x\notin L$ 时输出 1。\newline
		
		类 $\bf BPP$ 捕获了具有双边误差的概率算法。${\rm Pr}[M(x)=L(x)]\ge 2/3$
	\end{frame}

	\begin{frame}{单边误差,One-sided Error}
		单边误差的概率算法,即当 $x\notin L$ 时,算法不会输出 1,但仍可能在 $x\in L$ 时输出 0。\newline
		
		\includegraphics[width=\linewidth]{../../7/note.assets/image-20210509133757822.png}\newline
		
		有 $\bf RP\subseteq NP$。类 ${\bf coRP}=\{L|\overline{L}\in{\bf RP}\}$ 捕获“另一方向”的单边误差算法。
	\end{frame}

	\begin{frame}{零边误差}{Zero-sided Error}
		期望运行时间:对一个 PTM $M$ 及输入 $x$,随机变量 $T_{M,x}$ 为 $M$ 在 $x$ 上的运行时间,${\rm Pr}[T_{M,x}=T]=p$ 表示 $M$ 在 $x$ 上 $T$ 步内停机的概率是 $p$。如果对每个 $x\in\{0,1\}^*$ 期望 ${\rm E}[T_{M,x}]$ 最多为 $T(|x|)$,则称 $M$ 具有期望运行时间 $T(n)$。\newline
		
		\includegraphics[width=\linewidth]{../../7/note.assets/image-20210509153030302.png}\newline
		
		\includegraphics[width=\linewidth]{../../7/note.assets/image-20210509153059395.png}
	\end{frame}
	
	\section{定义的稳健性}
	\begin{frame}{精确常数的作用:减少误差}
		常数 $2/3$ 可以由任何大于 $1/2$ 的常数代替:\newline
		
		\includegraphics[width=\linewidth]{../../7/note.assets/image-20210509153425924.png}\newline
		
		通过若干次重复的主要结果作为输出,可以减小误差,提高结果的准确率。
	\end{frame}

	\begin{frame}{期望运行时间和最坏情况运行时间}
		定义 RTIME 和 BPTIME 时,要求机器在 T(n) 时间内停机,而不管它的随机选择,但是可以使用期望运行时间代替。\newline
		
		因为一个期望运行时间为 T(n) 的 PTM 可以变换为一个运行最多 100T(n) 步的 PTM,同时简单地增加一个计数器,在太多步之后以任意输出停止即可。而根据马尔可夫不等式,M 运行超过 100T(n) 步的概率最多为 1/100,因此这将使接受概率最多改变 1/100。
	\end{frame}

	\begin{frame}{允许更一般的随机选择}
		当生成0/1的生成器生成0的概率为$\rho\neq1/2$时,若$\rho$是可以高效计算(每一位可以多项式时间内计算)的数,则对应的概率算法的计算能力不会增加:\newline
		
		\includegraphics[width=\linewidth]{../../7/note.assets/image-20210509154955249.png}\newline
		
		相反的,$\rho\neq\cfrac{1}{2}$的概率算法的计算能力也不弱于标准概率算法。
	\end{frame}
	
	\section{随机规约}
	\begin{frame}{随机规约}
		\includegraphics[width=\linewidth]{../../7/note.assets/image-20210509155949766.png}\newline
		
		$\bf NP$ 可以定义为集合 $\{L:L\le_p{\rm 3SAT}\}$,替换为 $\le_r$ 可以得到 $\bf BP$ 版本的 $\bf NP$ 定义:
		
		\includegraphics[width=\linewidth]{../../7/note.assets/image-20210509162341964.png}
	\end{frame}

	\section{空间有界随机计算}
	\begin{frame}{类$\bf BPL$ 和 $\bf RL$}
		如果在任何计算分支中,使用的非空纸带最多为 $O(S(n))$,称一个 PTM 使用 $S(n)$ 的空间。类 $\bf BPL$ 和 $\bf RL$ 是类 $\bf L$ 的双边误差和单边误差的概率类似版本:\newline
		
		\includegraphics[width=\linewidth]{../../7/note.assets/image-20210509162758560.png}
	\end{frame}

	\begin{frame}{UPATH问题的$\bf RL$-算法}
		\includegraphics[width=\linewidth]{../../7/note.assets/image-20210509163216637.png}
		
		\begin{block}{UPATH}
			给定一个 $n$ 个顶点的无向图,以及两个顶点 $s$ 和 $t$,判定 $s$ 和 $t$ 是否相连。
		\end{block}
		
		算法:从 $s$ 开始随机走长度为 $l=100n^4$ 的路径。也即初始化变量 $v$ 为 $s$,每一步随机选择 $v$ 的一个邻居 $u$,然后将 $u$ 赋值给 $v$。当且仅当在 $l$ 步内到达 $t$ 时接受。\newline
		
		对数空间,只需要存储顶点的编号。若 $s$ 与 $t$ 不相连,算法不接受;若 $s$ 和 $t$ 相连,则从 $s$ 走到 $t$ 的期望步数最多为 $10n^4$,接受的概率最少为 $3/4$。
	\end{frame}
		
\end{document}